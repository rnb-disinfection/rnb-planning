\documentclass[10pt,a4paper]{article}
\usepackage[utf8]{inputenc}
\usepackage{amsmath}
\usepackage{amsfonts}
\usepackage{amssymb}
\usepackage{graphicx}
\usepackage{booktabs}
\usepackage{mathtools}
\usepackage[format=hang,justification=justified,labelfont=bf,margin={2.5cm,2.5cm}]{caption}

\usepackage[left=3.5cm,right=3.5cm,top=3cm,bottom=3cm]{geometry}
\author{Erwin Aertbeli\"en}
%date{17 December 2015}
\title{An Expressiongraph Node for Collision Checking}

\newcommand{\transf}[2]{ \prescript{#2}{#1}{\mathbf{T}} }
\newcommand{\rotm}[2]{ \prescript{#2}{#1}{\mathbf{R}} }
\newcommand{\position}[2] {\prescript{}{#1}{\mathbf{p}}^{#2}}
\newcommand{\velocity}[2] {\prescript{}{#1}{\mathbf{v}}^{#2}}
\newcommand{\rotvelmat}[2] {[\prescript{}{#1}{\boldsymbol{\omega}}^{#2} \times ]}
\newcommand{\rotvel}[2] {\prescript{}{#1}{\boldsymbol{\omega}}^{#2}}
\newcommand{\myvec}[1] {\mathbf{\boldsymbol{#1}}}

\setlength{\parindent}{0cm}

\setlength{\parskip}{0.3cm plus3mm minus1mm}

\begin{document}
\maketitle



\section{Introduction}
This document describes the collision library for expressiongraphs: \emph{expressiongraph\_collision}.  This library allows to express the distance between geometric objects.  This library uses the FCL library for 
the underlying geometric primitives and distance computations, and extends this to compute
not only the distance but also its Jacobian.

Only the distance between \emph{two} \emph{convex} objects is considered. Except for degenerate cases such as where planar faces are parallel to each other, we can define points of closest distance $p_a$ on object 1 and $p_b$ on object 2: points on one of the objects that are closest to the other object. 

To express the derivative of an orientation, rotational velocities are used, even when expressing partial derivatives towards other variables beside time.  The pose of object 1 is described $\transf{w}{1}$ by a rotation matrix $\rotm{w}{1}$ and an origin $\position{w}{1}$ The partial derivative of the pose of object 1 is described by a rotational velocity $\rotvel{w}{1}$ and a translational velocity $\velocity{w}{1}$.  Idem for the pose and partial derivative of the pose of object 2.


\section{Computation of the Jacobian}

If a point $p_i$ lies on a vertex, it will not change its location on the object after
infinitesimal movement of the object.  If a point $p_i$ lies on a surface, there is a supporting plane that is tangent to the surface and that is completely outside the object.
The (infinitesimal) movement of the contact point in this surface will not influence the
distance.  In other words, the derivative of the distance between two convex objects can
be computed by fixing the points of closest distance $p_1$ and $p_2$ to their objects; and computing the derivative of the distance between the points $p_1$ and $p_2$:
\begin{align}
d(\mathrm{obj}_1, \mathrm{obj}_2) = \left\Vert \mathbf{e} \right\Vert.
\end{align}
where $\mathbf{e}$ is:
\begin{align}
\mathbf{e} = \transf{w}{1}\position{w}{1} - \transf{w}{2}\position{w}{2}
\end{align}

The partial derivative $\partial \mathbf{e}/\partial q_i$ can be computed as:
\begin{align}
\frac{\partial \mathbf{e}}{\partial q_i} &=
\rotvelmat{w}{1}\rotm{w}{1}\position{1}{a} + \velocity{w}{1}
- \rotvelmat{w}{2}\rotm{w}{2}\position{2}{b} - \velocity{w}{2}
\end{align}
The derivative towards the distance is:
\begin{align}
\frac{\partial}{\partial q_i} d(\mathrm{obj}_1, \mathrm{obj}_2) =
\frac{\partial}{\partial q_i} \sqrt{ \mathbf{e}^T \mathbf{e}} =
-\frac{\mathbf{e}^T}{\sqrt{ \mathbf{e}^T \mathbf{e}}}\frac{\partial \mathbf{e}}{\partial q_i}
\end{align}

In the case where the the points $p_1$ and $p_2$ are not unique, such as with two parallel planes, the derivative of the distance does not exists: the directional derivative depends on the direction by which one approaches.  So, choosing one vertex is not a bad choice (or as bad a choice as any other choice).

\section{Superquadrics}

\subsection{GJK Algorithm, support function and supporting point}
The GJK algorithm~\cite{gilbert1988fast} is used for computing the distance between convex geometric objects.  A property of this algorithm is that it only needs the support function and the supporting point.  The convex geometric objects are described by a set $X$ of discrete points and its convex hull $co~X$.

The support function $h_X(\myvec{\eta})$ of a set $X$ in the direction $\myvec{\eta}$ is equal to
\begin{align}
h_X(\myvec{\eta}) = \mathrm{max}\left\lbrace  \myvec{x} \cdot \myvec{\eta} \mid \myvec{x} \in X \right\rbrace
\end{align}
A supporting point $s_X(\myvec{\eta})$ is a point on the surface such that
\begin{align}
h_X(\myvec{\eta}) = s_X(\myvec{\eta}) \cdot \myvec{\eta}
\end{align}

\subsection{Support function and supporting point of an implicit surface}

An implicit surface is described by the set of points $\mathbf{p}$ that satisfy the following equation:
\begin{align}
    f(\mathbf{p}) = 1
\end{align}


The following vector is normal to the implicit surface in the point $\mathbf{p}_0 = (x_0,y_0,z_0)$:
\begin{align}
    \mathbf{f}_n(\mathbf{p_0}) =
    \left.  
    \begin{bmatrix}
        \frac{\partial f}{\partial x} &
        \frac{\partial f}{\partial y} &
        \frac{\partial f}{\partial z} 
    \end{bmatrix} 
    \right|_{\mathbf{p_0}}
    %=
    %\begin{bmatrix}
    %    \frac{ k p_x^{k-1}}{a^k} &
    %    \frac{ k p_y^{k-1}}{b^k} &
    %    \frac{ k p_z^{k-1}}{c^k} 
    %\end{bmatrix}
\end{align}


If $\mathbf{p}_0$ is a supporting point, the normal to the support plane should be parallell to $\myvec{\eta}$:
\begin{align}
    \mathbf{f}_n(\mathbf{p}_0) = -\alpha \myvec{\eta},
    \label{eq:norm}
\end{align}
with $\alpha$ an arbitrary positive constant, and the point $\mathbf{p}_0$ should be on the surface:
\begin{align}
    f(\mathbf{p_0}) = 1.
    \label{eq:surf}
\end{align}
The above results in 4 equations ([\ref{eq:norm}] and [\ref{eq:surf}]) in 4 unknowns ( $\mathbf{p}_0$ and $\alpha$ ).

\subsection{Ellipsoid}

\subsection{Superquadrics}
A superquadric is a convex geometric object satisfying the following equation:
\begin{align}
    \left( \frac{p_x}{a} \right)^{k_1} +
    \left( \frac{p_y}{a} \right)^{k_2} +
    \left( \frac{p_z}{a} \right)^{k_3}    = 1.
\end{align}
For the purpose of this document $k_1 = k_2 = k_3 = k$ and $k$ is even.
%
%

%In the above formula, the factor $k$ can be dropped because it uniformly scales
%the normal.
%
%The normal to the support plane should be parallell to $\myvec{\eta}$ 
%\begin{align}
    %\mathbf{f}_n(\mathbf{p}) = \alpha \myvec{\eta}
%\end{align}
%
%\begin{align}
    %p_x^{k-1} =
    %++
%
%
%
%
%
%










%Since a superquadric is strictly convex, the supporting point is unique and the
%normal of the supporting plane should be parallel to the direction $\myvec{\eta}$: there exists a scaling factor $\alpha$ such that:
%\begin{align}
%\myvec{\eta}_x &= \alpha \frac{n}{a} \mathrm{sgn}^n(x) \left| \frac{x_0}{a} \right|^{n-1} \\
%\myvec{\eta}_y &= \alpha \frac{m}{b} \mathrm{sgn}^m(y) \left| \frac{y_0}{b} \right|^{m-1}\\
%\myvec{\eta}_z &= \alpha \frac{k}{c} \mathrm{sgn}^k(z) \left| \frac{z_o}{c} \right|^{k-1}
%\end{align}
%
%\begin{align}
%x_0 = \alpha^{-\frac{1}{n-1}} 
%\left( \frac{1}{n} a^n \myvec{\eta}_x \mathrm{sgn}^n(x_0)  \right)^{\frac{1}{n-1}}  
%\end{align}
%
%
%\begin{align}
%x_0 = \alpha^{-\frac{1}{n-1}} a
%\left( \frac{1}{n} a \myvec{\eta}_x \mathrm{sgn}^n(x_0)  \right)^{\frac{1}{n-1}}  
%\end{align}
%
%\begin{align}
%\alpha^{-\frac{n}{n-1}} 
%\left( \frac{1}{n} a \myvec{\eta}_x \mathrm{sgn}^n(x_0)  \right)^{\frac{n}{n-1}}  
%\end{align}
%
%
\section{Available primitives}

See table \ref{table:primitives}.

\begin{table}
	\centering
	\begin{tabular}{@{}lp{10cm}@{}}
	\toprule
	Primitives & Description\\
	\midrule
	Box & a box with size $L_x$ in the $x$-direction, size $L_y$ in the $y$-direction and size $L_z$ in the $z$-direction,
	      centred around the middle point of the box.\\
	Sphere & a sphere with radius $R$, centred around its middle point.\\
	Cylinder & a cylinder with length $L$ and radius $R$. \\
	Capsule & a cylinder with length $L$ and radius $R$, capped at each end with a half sphere with radius $R$.\\
    Cone & a cone with radius $R$ of the base and height $L$\\
    ConvexShape & an arbitrary convex polyhedron, specified using the Wavefront file format.\\
	\bottomrule
	\end{tabular}
	
	\caption{Table of allowable primitives. The distances between any two primitives can be computed.}
    \label{table:primitives}
\end{table}


\bibliography{collision}{}
\bibliographystyle{plain}
\end{document}
\end{document}
